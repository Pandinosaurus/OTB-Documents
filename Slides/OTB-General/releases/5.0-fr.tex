\begin{frame}
\frametitle{5.0 (Mai 2015)}
\begin{block}{Modularité}
\begin{itemize}
\item Une meilleure organisation du code, en modules cohérents (124 modules et
    16 groupes) contenants sources, tests et applications.
\item Gestion des dépendances
\item Contributions externes: \url{https://www.orfeo-toolbox.org/external-projects/}
\end{itemize}
\end{block}

\begin{block}{SuperBuild}
\begin{itemize}
\item Il n'y a plus de logiciels tiers dans l'OTB
\item Le Superbuild, télécharge, configure, compile et installe les dépendances
\item Il existe également un mode \textit{offline} pour compiler l'OTB sans
  accès internet (en avion par exemple).
\end{itemize}
\end{block}
\end{frame}

\begin{frame}
\frametitle{Gouvernance ouverte: Project Steering Committee}
\begin{block}{Genèse du PSC}
  \begin{itemize}
  \item Jusqu'en 2015: l'OTB, un logiciel à sources ouvertes
  \item En mars 2015: l'OTB devient un logiciel libre, le CNES nomme un PSC initial
  \end{itemize}
\end{block}
\begin{block}{Un club de développeurs, pas de décideurs}
  \begin{itemize}
  \item Pilotage haut niveau du projet, roadmaps, communication, planification
  \item Vote les RFCs: Tous les membres ont le même poids dans les votes ($\pm 1$, $\pm 0$)
  \item Les sièges n'expirent pas, sortie par démission ou vote d'expulsion
  \item Le PSC n'est pas une entité légale et n'a pas de moyens propres
  \end{itemize}
\end{block}
\begin{block}{En chiffres}
  \begin{itemize}
  \item 5 membres de 4 entités différentes
  \item 2 release sous l'égide du PSC (5.2, 5.4)
  \item 3 meetings en ligne (logs publics)
  \end{itemize}
\end{block}
\end{frame}
